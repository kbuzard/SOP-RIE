\documentclass[12pt]{report}

\addtolength{\textwidth}{1in}
\addtolength{\oddsidemargin}{-.5in} %left margin
\addtolength{\evensidemargin}{-.5in}
\setlength{\parindent}{0pt}
\setlength{\textheight}{8.5in}
\setlength{\topmargin}{-.5in}
\setlength{\headheight}{18.0pt}
\setlength{\footskip}{.375in}
\renewcommand{\baselinestretch}{1.0}
\linespread{1.2}
\usepackage{fancyhdr}
\pagestyle{fancy}
\renewcommand{\headrulewidth}{1pt}
\lhead{Trade Agreements in the Shadow of Lobbying}
%\chead{}
\rhead{Kristy Buzard}
%\fancyfoot[C]{}
%\fancyhead[R]{}

\usepackage[pdftex,
bookmarks=true,
bookmarksnumbered=false,
pdfview=fitH,
bookmarksopen=true]{hyperref}
\usepackage[pdftex]{graphicx}
\setcounter{secnumdepth}{-1}


\usepackage[usenames,dvipsnames]{color}
\usepackage{cite}
\usepackage{times, verbatim,bm,pifont}
\usepackage{amsbsy,amssymb, amsmath, amsthm, MnSymbol,bbding}


\begin{document}

\begin{center}
\large Response to Editor and Referee \\
\normalsize Month ???, 2016
\end{center}

\vskip.2in
Thank you. \\

Logistics. Points were not numbered, so pasting them here\\

\textit{The main issue is still point 2 of my previous report: the paper is fundamentally quite interesting but poorly executed. What I would like you to do is write a much more focused, crispier version of the paper. The idea is sufficiently straightforward that it does not need 28 pages of exposition. Please reduce the length of the paper by half.}

\vskip.3in
Stylistic advice:

\begin{itemize}
	\item Avoid changes of notation as you go along. This is totally unhelpful to the reader. Please pick one and stick to it throughout the paper.
	\item Avoid digressions, lengthy discussions, and overreach: you stretch your model far too much. Stick to the main story. The paper still degenerates far too often into a series of discussions that bring little additional insight.
	\item Section 2.4 either belongs to section 1.1 or should move as section 1.2. Section 2.4 is far too long.
	\item Your recall stuff far too often. Not all readers suffer from memory losses.
\end{itemize}


\vskip.3in
Some comments on substance, chronologically:
\begin{itemize}
\renewcommand\labelitemi{-}
	\item Assuming that lobbies represent only import-competing interests is a strong assumption and it requires theoretical and/or empirical justification.
	\item Your model is static, so claiming that it rationalizes the four-year delay and eventual ratification of the US-SK FTA is a stretch. In particular, the interactions with automakers suggests a that a bargaining game under incomplete information would be more appropriate.
  \item Your assumption that the identity of the median voter is not known a priori is standard in probability voting models, so you may want to stress the analogy between your setting and such models.
  \item \textit{You claim that trade agreements act as a domestic political commitment device. To make this point formally, it would be nice to establish conditions under which, in equilibrium, “fast track authority” dominates cherry picking ex ante (in a version of your model with several import competing sectors).}
	
  \item I don’t see how the move of Nature (which determines the probabilities o and o* is consistent with the idea that there is uncertainty about who is the median voter (which was part of your motivation). Assuming that o and o* are mutually exclusive is not great. It would be better to assume some timing (rather than have a simultaneous choice). Since the game is symmetric, assuming that Home’s legislature moves before Foreign’s would be without further loss of generality.
  \item “the uncertainty about the median legislator’s identity is resolved.” What is this uncertainty really about?
  \item The equilibrium selection criterion requires justification and motivation.
  \item \textit{Assumption 2 does not formalize any intuition: it is just a labelling.} \\
	Quite right. The text has been modified appropriately.
  \item I feel uneasy about Assumption 3. I would rather have it as an outcome and, if it is not the only possibly one, a discussion of the conditions under which this arises.
  \item The full separability of sectors implies very minimal interactions already. Assuming a fully symmetric model is then highly restrictive. How difficult would it be to relax the latter assumption?
  \item \textit{Expression 6 establishes that the necessary second order condition holds by assumption 1, does not it?} \\
	I agree that this is the case. As you so helpfully pointed this out in your last set of comments, I have included an argument to this effect in the two full sentences following Expression 6. If there is a more direct argument, I would be happy to replace the current one.
  \item Writing that lobbying expenditures are equal to 0.0007 is meaningless absent units. Are these cents? Trillions of dollars? It would be much more informative if these were expressed as a fraction of equilibrium profits.
  \item \textit{The last paragraph of section 4 is quite a stretch; is there any evidence in the political science literature that the separation of power is correlated with trade openness among democracies?} \\
	The issue is quite a nuanced, in particular among democracies. I have discussed the empirical evidence on democracies---separate from the claim that I made about moving from autocracy to democracy---from Henisz and Mansfield (2006) in a new paragraph. I have also added a reference to Frye and Mansfield (2003) to support the original claim. I don't consider any of this essential and would be happy to remove either or both paragraphs.
\end{itemize}


Best wishes,\\
\\
\\
Kristy Buzard

\end{document}