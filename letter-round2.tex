\documentclass[12pt]{report}

\addtolength{\textwidth}{1in}
\addtolength{\oddsidemargin}{-.5in} %left margin
\addtolength{\evensidemargin}{-.5in}
\setlength{\parindent}{0pt}
\setlength{\textheight}{8.5in}
\setlength{\topmargin}{-.5in}
\setlength{\headheight}{18.0pt}
\setlength{\footskip}{.375in}
\renewcommand{\baselinestretch}{1.0}
\linespread{1.2}
\usepackage{fancyhdr}
\pagestyle{fancy}
\renewcommand{\headrulewidth}{1pt}
\lhead{Trade Agreements in the Shadow of Lobbying}
%\chead{}
\rhead{Kristy Buzard}
%\fancyfoot[C]{}
%\fancyhead[R]{}

\usepackage[pdftex,
bookmarks=true,
bookmarksnumbered=false,
pdfview=fitH,
bookmarksopen=true]{hyperref}
\usepackage[pdftex]{graphicx}
\setcounter{secnumdepth}{-1}


\usepackage[usenames,dvipsnames]{color}
\usepackage{cite}
\usepackage{times, verbatim,bm,pifont}
\usepackage{amsbsy,amssymb, amsmath, amsthm, MnSymbol,bbding}


\begin{document}

\begin{center}
\large Response to Editor \\
\normalsize June 2016
\end{center}

\vskip.2in
Thank you for the second opportunity to bring this paper above your bar. I believe the paper is now much, much tighter and better organized. Below I detail how I responded to each of your points. As they were not numbered, I have listed each of your points in order (italicized) with the corresponding response immediately below. Because the changes in the text are so dramatic, bolding every one became nearly meaningless. I therefore bolded only those passages that have changed in response to a specific suggestion. \\

\textit{The main issue is still point 2 of my previous report: the paper is fundamentally quite interesting but poorly executed. What I would like you to do is write a much more focused, crispier version of the paper. The idea is sufficiently straightforward that it does not need 28 pages of exposition. Please reduce the length of the paper by half.} \\
I believe I have achieved what you have asked for through a combination of jettisoning superfluous material, moving essential material so that it was not repeated, and generally tightening the exposition. I've been able to cut the length of the body of the paper from 28 pages to $18\frac{1}{2}$.  


\vskip.3in
Stylistic advice:

\begin{itemize}
\renewcommand\labelitemi{-}
	\item \textit{Avoid changes of notation as you go along. This is totally unhelpful to the reader. Please pick one and stick to it throughout the paper.} \\
	I have worked hard to standardize notation. I would welcome any suggestions about remaining inconsistencies. The one potential duplication of which I am aware is using ``PS'' (producer surplus) and $\pi$ in the government welfare functions and lobby's objective function respectively although they represent the same quantity. This is a purposeful stylistic choice but it can be reversed to use $\pi$ throughout.
	\item \textit{Avoid digressions, lengthy discussions, and overreach: you stretch your model far too much. Stick to the main story. The paper still degenerates far too often into a series of discussions that bring little additional insight.} \\
	I have ruthlessly excised all of the above, leaving only those that were added as a result of the refereeing process at this journal or that have arisen consistently in the past.
	\item \textit{Section 2.4 either belongs to section 1.1 or should move as section 1.2. Section 2.4 is far too long.} \\
	The text of Section 2.4 has been shorted and redistributed. I inserted the few paragraphs that concern assumptions into Section 2.2 (preferences) since they could not come before the model was introduced. All the rest that I deemed essential have been integrated in condensed form into the introduction and literature review. I have bolded them where inserted.
	
	\item \textit{Your recall stuff far too often. Not all readers suffer from memory losses.} \\
	All recalls have been removed as far as I can tell.
\end{itemize}


\vskip.3in
Some comments on substance, chronologically:
\begin{itemize}
\renewcommand\labelitemi{-}
	\item \textit{Assuming that lobbies represent only import-competing interests is a strong assumption and it requires theoretical and/or empirical justification.} \\
	Done, on page 6. Footnote 6 explains that extending to multiple lobbies is possible.
	\item \textit{Your model is static, so claiming that it rationalizes the four-year delay and eventual ratification of the US-SK FTA is a stretch. In particular, the interactions with automakers suggests a that a bargaining game under incomplete information would be more appropriate.} \\
	I have removed this claim.
  \item \textit{Your assumption that the identity of the median voter is not known a priori is standard in probability voting models, so you may want to stress the analogy between your setting and such models.} \\
	Thank you for this suggestion. I have incorporated a reference on page 2.
	
  \item \textit{You claim that trade agreements act as a domestic political commitment device. To make this point formally, it would be nice to establish conditions under which, in equilibrium, ``fast track authority'' dominates cherry picking ex ante (in a version of your model with several import competing sectors).} \\
		I believe that I have already made this point formally in Result 3. The proof invokes Result 2 and Lemma 3 as it combines both dynamics that are explained intuitively in the text. I have made this more explicit by referring to those results in the text.
			
		I agree that the idea you propose concerning fast-track authority / TPA is an important one and it is an idea that a colleague and I are exploring in a separate project as I believe it is beyond the scope of the current paper. The extension of the model to one with multiple sectors and lobbies is one that is worthwhile for many purposes. However, the essential idea of the current paper is, as you have judged, `sufficiently straightforward' that in my view the complications involved in the multi-sector model are not warranted.
	
  \item \textit{I don't see how the move of Nature (which determines the probabilities o and o* is consistent with the idea that there is uncertainty about who is the median voter (which was part of your motivation). Assuming that o and o* are mutually exclusive is not great. It would be better to assume some timing (rather than have a simultaneous choice). Since the game is symmetric, assuming that Home’s legislature moves before Foreign's would be without further loss of generality.} \\
	I view the move of nature that determines the probabilities $o$ and $o^*$ as distinct from the uncertainty about the identity of the median legislator. This is really just a theoretical device to simplify the model.
	
	The sequential model you suggested is very attractive. I worked it out and wrote it into the draft. All the results go through (with one small additional assumption on magnitudes: it must be ensured that the lobby's direct response to an increase in trade agreement tariffs is stronger than a new effect wherein the lobby might increase effort to compensate for the trading partner's legislature reducing its break probability in response to that same tariff increase). However, the model is significantly complicated as two additional actors have to be modeled and results added since the foreign lobby and legislature can no longer be accounted for by symmetry. This added considerable length without compensating insight, so I opted to add a footnote explaining the near-equivalence of the models in line with your desire to simplify the presentation and not distract from the central point. The new text reads as follows can can be found in Footnote 10 on Page 7:
	\begin{quote}
		If the break decisions in the two countries are sequential with the executives behind a veil of ignorance about which legislature moves first, the end results are the same but additional intermediate results are needed.
	\end{quote}
	
  \item \textit{``the uncertainty about the median legislator's identity is resolved.'' What is this uncertainty really about?} \\
	This is now explained on page 2, so I did not repeat it or reference it here in line with the earlier stylistic advice.
	
  \item \textit{The equilibrium selection criterion requires justification and motivation.} \\
	Done. See page 9.
	
  \item \textit{Assumption 2 does not formalize any intuition: it is just a labelling.} \\
	Quite right. The text has been modified appropriately. This is now near the top of page 8.
	
  \item \textit{I feel uneasy about Assumption 3. I would rather have it as an outcome and, if it is not the only possibly one, a discussion of the conditions under which this arises.} \\
	I can remove Assumption 3, but the following happens: we get a second case for most of the results for when there are extreme realizations of $\theta$ such that the median legislator is more pro-trade than the executive at the trade-war phase so that the trade war tariffs are below the trade agreement tariffs. In this case, the results of the paper are uninteresting. So removing this assumption complicates the analysis and results without adding anything interesting or particularly realistic.
	
	Alternatively, I can make an assumption on the support of $\theta$ and the functional form of $\gamma$ so that Assumption 3 becomes a result, but this seems overly restrictive to make the points of the paper.
	
  \item \textit{The full separability of sectors implies very minimal interactions already. Assuming a fully symmetric model is then highly restrictive. How difficult would it be to relax the latter assumption?} \\
	It's not very hard to relax the symmetry of the \textit{equilibrium}, but it requires adding some restrictions so that the impact on the break probability from $\tau^{*a}$ in Lemma 2 does not outweigh that of $\tau^a$ in Lemma 1. I chose symmetry because it's prevalent in the literature (along with separability) and avoids such seemingly ad-hoc assumptions. I also take advantage of the symmetry for the argument establishing existence of the executives' problem and I'm not completely sure the most elegant way with the least additional restrictions to get that without symmetry.
	
	How hard it would be to make the underlying model asymmetric depends on the particular form of asymmetry. The one such avenue that is regularly followed in this literature is to have political pressure in only one country. That's a simplification so would easily go through. The only other asymmetry I've seen with any regularity is in the size of the economies (one large, one small). I think I might need a restriction on the size of the asymmetry in this case. I'd also have to generalize the bargaining solution in the case of any model asymmetry.
	
  \item \textit{Expression 6 establishes that the necessary second order condition holds by assumption 1, does not it?} \\
	I agree that this is the case. As you so helpfully pointed this out in your last set of comments, I have included an argument to this effect in the two full sentences following Expression 6. If there is a more direct argument, I would be happy to replace the current one.
  \item \textit{Writing that lobbying expenditures are equal to 0.0007 is meaningless absent units. Are these cents? Trillions of dollars? It would be much more informative if these were expressed as a fraction of equilibrium profits.} \\
	I have added the numbers both for expenditure as a fraction of profits and for net profits.
	
  \item \textit{The last paragraph of section 4 is quite a stretch; is there any evidence in the political science literature that the separation of power is correlated with trade openness among democracies?} \\
	The issue is quite a nuanced, in particular among democracies. In line with the earlier suggestion to remove digressions, I removed the paragraph at the end of section 4 regarding moves from autocracy to democracy. I have, however, added a short discussion regarding democracies citing from Henisz and Mansfield (2006) at the end of the introduction to support the argument that the model is broadly applicable.
\end{itemize}

\vskip.3in
Best wishes,\\
\\
\\
Kristy Buzard

\end{document}