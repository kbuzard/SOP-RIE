\documentclass[12pt]{article}

\addtolength{\textwidth}{1in}
\addtolength{\oddsidemargin}{-.5in} %left margin
\addtolength{\evensidemargin}{-.5in}
\setlength{\textheight}{8.5in}
\setlength{\topmargin}{-.5in}
\setlength{\headheight}{18.0pt}
\setlength{\footskip}{.375in}
\renewcommand{\baselinestretch}{1.0}
\linespread{1.2}
\usepackage{fancyhdr}
\pagestyle{fancy}
\renewcommand{\headrulewidth}{1pt}
\lhead{Trade Agreements in the Shadow of Lobbying}
%\chead{}
\rhead{Kristy Buzard}
%\fancyfoot[C]{}
%\fancyhead[R]{}

\usepackage[pdftex,
bookmarks=true,
bookmarksnumbered=false,
pdfview=fitH,
bookmarksopen=true]{hyperref}
\usepackage[pdftex]{graphicx}
\setcounter{secnumdepth}{-1}


\usepackage[usenames,dvipsnames]{color}
\usepackage{cite}
\usepackage{times, verbatim,bm,pifont}
\usepackage{amsbsy,amssymb, amsmath, amsthm, MnSymbol,bbding}


\begin{document}

\begin{center}
\large Response to Editor and Referee \\
\normalsize February 2, 2016
\end{center}

Editor's point $\#$4: I have added a traded numeraire good to address the concern about balanced trade. As all the results hinge on the incentives of the lobby, the executive and the median legislator---none of which are impacted by the addition of the numeraire good---this does not change any of the results. In carefully examining this issue, I realized that it would be more complete to include labor income in the welfare functions of the executive and lobby (again, this would change no results, as they all involve either differencing two welfare terms or taking derivatives with respect to a tariff). However, since much of the literature I'm following abstracts from a balanced trade condition and a numeraire sector with labor input, this would complicate the exposition and comparison to the extant literature. I have therefore chosen to insert Footnote [{\color{blue}ADD FOOTNOTE REFERENCE}] to this effect. It would be a simple matter to put labor income into the welfare functions if that is judged preferable.

Editor's point $\#$5: I have added expectations operators to Equations (3) and (4) [{\color{blue}CHECK EQ:JV AND EQ:LV}] to address the redundant notation. Thank you for pointing out this needed clarification.

Editor's point $\#$7: Thank you for pointing out that the second order condition holds everywhere and that the proof of what was formerly Lemma 1 can be drastically simplified. I moved the discussion of the second order condition into the text and integrated it with the simple result on the increasingness of trade war tariff in the political economy weight. As this did not need to be referenced elsewhere and no longer requires a proof in the appendix, it is no longer labeled Lemma 1. Likewise, Lemma 0 is no longer necessary and has been removed. Note that this advances the numbering of the lemmas in the revised draft relative to the previous draft. 

Editor's point $\#10$: I have removed the extensions (Section 6) per your suggestion. {\color{blue}ADD discussion about fixing exposition around one break at a time}


\end{document}