\documentclass[12pt]{report}

\addtolength{\textwidth}{1in}
\addtolength{\oddsidemargin}{-.5in} %left margin
\addtolength{\evensidemargin}{-.5in}
\setlength{\parindent}{0pt}
\setlength{\textheight}{8.5in}
\setlength{\topmargin}{-.5in}
\setlength{\headheight}{18.0pt}
\setlength{\footskip}{.375in}
\renewcommand{\baselinestretch}{1.0}
\linespread{1.2}
\usepackage{fancyhdr}
\pagestyle{fancy}
\renewcommand{\headrulewidth}{1pt}
\lhead{Trade Agreements in the Shadow of Lobbying}
%\chead{}
\rhead{Kristy Buzard}
%\fancyfoot[C]{}
%\fancyhead[R]{}

\usepackage[pdftex,
bookmarks=true,
bookmarksnumbered=false,
pdfview=fitH,
bookmarksopen=true]{hyperref}
\usepackage[pdftex]{graphicx}
\setcounter{secnumdepth}{-1}


\usepackage[usenames,dvipsnames]{color}
\usepackage{cite}
\usepackage{times, verbatim,bm,pifont}
\usepackage{amsbsy,amssymb, amsmath, amsthm, MnSymbol,bbding}


\begin{document}

\begin{center}
\large Response to Editor and Referee \\
\normalsize February 2, 2016
\end{center}

\vskip.2in
I would like to begin by thanking you both for very thoughtful and helpful suggestions on the previous draft of the paper. I believe that each one contributed to improving the current draft. In particularly, following your comments, the first half of the paper has been almost entirely re-written. \\

To make the review process as efficient as possible, I have bolded the changes made as requested. Note that I have only bolded text when it was newly-written or changed, not when it was moved, as were substantial portions of Sections 1 and 2. This letter details all changes beyond correction of typos and minor expository changes. I have tried as much as possible to list the responses here in the order the changes appear in the text. \\

\textbf{Editor's point $\#$1 / Referee's point $\#$4}: I have revised the introduction according to the formula you suggested. I have toned down considerably the claims about the PFS puzzle as both you and the referee suggest; they are now at the very end of the introduction and in a soft way in the abstract. \\

\textbf{Referee's point $\#$1} (``Introduction is derailing at the start''): I agree and I thank you for your detailed suggestions about how to reformulate the introduction and incorporate a toned-down discussion about the PFS puzzle. Please see response to Editor's point $\#$1 for logistics. \\

\textbf{Referee's point $\#$2} (``Intuition in the Introduction about political uncertainty''): There are several kinds of intuition about political uncertainty, and I struggled a bit to guess which kind was meant. Because I already give some intuition about the effects of political uncertainty on the results, I went with the interpretation that this meant intuition/interpretation about what I mean by political uncertainty itself. I have thus added a few sentences in the fifth paragraph of the introduction to this end. I would be happy to do something different/additional if I have interpreted this suggestion incorrectly. \\

\textbf{Editor's point $\#$8}: I agree that my model does not nest PFS. The discussion in the new Section 2.4, footnote 17 and Section 5.2 is meant to highlight the similarities and differences between my model and the PFS model. I hope that I have addressed your concerns about over-selling the contribution relative to the PFS puzzle so that this is no longer an issue. \\

\textbf{Editor's point $\#$4}: I have added a traded numeraire good to address the concern about balanced trade. As all the results hinge on the incentives of the lobby, the executive and the median legislator---none of which are impacted by the addition of the numeraire good---this does not change any of the results. In carefully examining this issue, I realized that it would be more complete to include labor income in the welfare functions of the executive and lobby (again, this would change no results, as they all involve either differencing two welfare terms or taking derivatives with respect to a tariff). However, since much of the literature I'm following abstracts from a balanced trade condition and a numeraire sector with labor input, this would complicate the exposition and comparison to the extant literature. I have therefore chosen to insert Footnote 14 to this effect. It would be a simple matter to put labor income into the welfare functions if that is judged preferable. \\

\textbf{Editor's point $\#$2}: I have created four subsections within Section 2 to structure the exposition of the model. I think this is much clearer than the previous draft. Note that I have moved (former) Assumptions 2 and 3 from Section 3 into Section 2.4. They are now labeled Assumptions 1 and 2, while the former Assumption 1 is now Assumption 3 because of this rearrangement. \\

\textbf{Editor's point $\#$3}: Thank you for the suggestion to add an extensive form and to define the game more clearly and formally. I think the addition of the normal form as Figure 1 made this portion of the paper significantly more readable. I hope you will agree. The discussion of the players, timing, and the presentation of the extensive form are at the end of Section 2.1 (The Basic Setup). I largely allow the extensive form to do the work of laying out the strategy spaces so as not to clutter the text. The payoffs are gathered together in their own section (2.2: Preferences). \\

\textbf{Editor's point $\#$9}: I have added the move of nature that determines whether the legislature gets the opportunity to break the agreement to the extensive form as well as to the discussion of the timing in the initial discussion of the model (Section 2.1: The Basic Setup). I have gathered all relevant discussion in this section as suggested. \\

\textbf{Editor's point $\#$5}: I have added expectations operators to Equations (2) and (3) to address the redundant notation. Thank you for pointing out this needed clarification. \\

\textbf{Editor's point $\#$6}: I have rewritten the passage about the influence of lobbying on the executive to read
\begin{quote}
  This setup does not strictly require that the executives are not lobbied. The assumption that the weight the executives place on import-competing profits is constant only requires that the extent to which they favor this industry cannot be \textit{directly} influenced by lobbying effort. The assumption is innocuous as long as the executives' preferences are not directly altered in a significant way by lobbying over trade.
\end{quote}
I think this is a softer claim than what was in the earlier draft and avoids getting into an interpretation of exactly what `Protection for Sale' means in this context since this is not central---what is important is to defend the modeling assumption directly.
I have also removed the duplicate statement of Assumption 1 and combined redundant text surrounding the two statements. I have added $\forall e$ to the statement of Assumption 1, which is now relabeled Assumption 3 because I have moved the former Assumptions 2 and 3 earlier in Section 2.4 as part of the restructuring discussed in point $\#2$. \\

\textbf{Referee's point $\#$3} (``Parliamentary democracies''): I have added a paragraph at the very end of Section 2 to address this concern. I would also note, as a point of interest, that the number of veto points does not always correlate particularly well with the political system. I was not easily able to find empirical work on the number of veto points on trade issues for Turkey, but Sparkes, Bump and Reich (2015) were somewhat surprisingly able to identify \textit{seven} veto points for health issues in Turkey.\footnote{Sparkes, S.P., Bump, J. B. and M. R. Reich (2015), ``Political Strategies for Health Reform in Turkey: Extending Veto Point Theory,'' \textit{Health Systems and Reform}, 1(4), 263-275.} \\

\textbf{Editor's point $\#$7}: Thank you for pointing out that the second order condition holds everywhere and that the proof of what was formerly Lemma 1 can be drastically simplified. I moved the discussion of the second order condition into the text (Section 3.1) and integrated it with the simple result on the increasingness of trade war tariff in the political economy weight. As this did not need to be referenced elsewhere and no longer requires a proof in the appendix, it is no longer labeled Lemma 1. Likewise, Lemma 0 is no longer necessary and has been removed. Note that this advances the numbering of the lemmas in the revised draft relative to the previous draft; this should not cause too much trouble as there were no changes to any of the results aside from the statement of Result 2 (see below). \\

\textbf{Editor's point $\#$10}: I have removed the extensions (Section 6) per your suggestion. Some of the discussion about the assumption that the legislatures do not take simultaneous decisions to break the agreement was moved into Section 2.1. \\


I made two other small changes:
\begin{itemize}
	\item I have removed the statement ``and the distribution of $\gamma(e,\theta)$ is weakly increasing in $e$'' from Result 2 because this is the content of Assumption 2 and does not need to be repeated.
	\item I have added some newer references to the uncertainty and trade agreements literature to the literature review and updated working papers where possible.\\
\end{itemize}

Again, I thank you for your thoughtful and thorough feedback and look forward to your further comments. \\

Best wishes,\\
\\
\\
Kristy Buzard

\end{document}